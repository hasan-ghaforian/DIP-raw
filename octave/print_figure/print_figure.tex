\documentclass{article}


\usepackage{fancyvrb}


\begin{document}
    

Suppose we ran these commands in \textit{Octave}:
\begin{Verbatim}
    > mu=linspace(-20,20,1000);
    > w=0.5;
    > y=w*sin(pi*mu*w)./(pi*mu*w);
    > hf = figure ();
    > plot(mu,y);
\end{Verbatim}
And you want to export the diagram to a `.tex' document. Now you have three
choices; but I prefer using \texttt{-dtikz} option:
\begin{itemize}
\item https://github.com/matlab2tikz/matlab2tikz
\item export plot data and text data separately:
    \begin{itemize}
    \item export figure as `eps':
    \begin{Verbatim}
    > print -depslatex "figure-fig.tex"
    \end{Verbatim}
    In this case you will get an `eps' file with name
    `figure-fig-inc.eps' which contains the figure plot and a `.tex'
    file with name `figure-fig.tex' which includes text, words and
    ... and also calls `.eps' file.
    \item export figure as `pdf' and import to your document:
    Then print to file with one of these:
    \begin{Verbatim}
    > print (hf, "plot15.pdf", "-dpdflatexstandalone");
    \end{Verbatim}
    In this case you will get a `pdf' file with name `plot15-inc.pdf'
    which contains the figure plot and `.tex' file with name
    `plot15.tex' which includes text, words and ... and also calls
    `.pdf' file.
    \end{itemize} 
    In this case, you can import plot and text data by placing content of
    `.tex' file in your main document or using \verb"\input" command for
    inputting that `.tex' file. \textbf{Note} that you have to update the
    path to `.pdf' or `.eps' files in \verb"\includegraphics" command of
    `.tex' file relative to your main `.tex' document. For example if you
    write this for inputting `.tex' file:
    \verb"\input{./figures/figure-fig.tex}" which means the inputted
    `.tex' file is in `./figures' directory, then in it the path to graphics
    must be \verb"\includegraphics[scale=1]{./figures/figure-fig-inc}".
\item using \texttt{-dtikz} option:\\
    This command will export all data of plot, text and ... to one `.tex'
    file. For example this command:
    \begin{Verbatim}
        > print -dtikz "myfile.tex"
    \end{Verbatim}
    will generate `myfile.tex' file for the \textit{current} figure and you only
    need to import it by using \verb"\input" command.
\end{itemize}

\end{document}