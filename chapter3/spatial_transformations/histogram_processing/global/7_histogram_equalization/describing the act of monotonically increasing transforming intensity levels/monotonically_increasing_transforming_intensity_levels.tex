\documentclass[a4paper]{article}

\usepackage{fancyvrb}
\usepackage{xcolor} % enables using colours
\usepackage{amsmath,amsthm,amssymb}
\usepackage{amsfonts}
\usepackage{alltt} %"alltt" en­vi­ron­ment which is like the ver­ba­tim
                    %en­vi­ron­ment ex­cept that \ and braces have their usual
                    %mean­ings
\usepackage{tikz}
\usepackage{cancel}
\usetikzlibrary{positioning}
%\usepackage{xepersian}

\pagecolor{black}
\color{white}

\begin{document}

\paragraph*{Goal}
is finding a transformation for each random variable to get a uniform random
variable.
\paragraph*{Way} is:
\begin{itemize}
    \item preface
        \begin{itemize}
            \item definition of PDF and CDF
            \item concept of differentiation of CDF and the PDF
            \item CDF of a random variable and its map under a monotonically increasing
            transformation 
        \end{itemize}
    \item proposing a special transformation
\end{itemize}

\pagebreak

\section{Random variable, CDF and PDF}

\paragraph*{Random Variable} is a function like $f$ which:

$$f: S \rightarrow \mathbb{R}$$

\paragraph*{Probability function} for random variable $Pr(B)$: Sum of probabilities for each
member of $B$.
For specific case $Pr(X=x_0) = Pr(x_0)$.

\paragraph*{Cumulative distribution function (CDF)}

\begin{itemize}
    \item for both discrete and continuous 
    \item its nature is Probability
\end{itemize}

\paragraph*{Probability density function (PDF)}{
    \begin{itemize}
        \item only for continuous random variables
        \item because $Pr$ for continuous random variables is $0$
        \item help us to calculate the $Pr$
        \item nature of PDF is \textbf{not} probability
            \begin{itemize}
                \item $pdf\ dx$ is of type probability
                \item there may be times that $PDF > 1$; but the role of $dx$
                will cause that not problematic
            \end{itemize}
        \item 
        
    \end{itemize}
}

\begin{align*}
    CDF(a) = Pr (X < a) \\
    = \int_{-\infty}^{a} pdf\ dx
\end{align*}

\begin{align*}
    pdf = \frac{d(CDF)}{dx}
\end{align*}

\begin{align*}
    Pr(a < X < b) = \int_{a}^{b} pdf\ dx
\end{align*}

\pagebreak

\section{Continuous functions and differentiation}

\paragraph*{}
If $f$ is continuous then:
\begin{itemize}
    \item {
        \begin{align}
            \exists x_0 \in [a,b] : \int_{a}^{b} f(x)\ dx = f(x_0) (b - a)
        \end{align}
    }     
    \item {
        \begin{align}
            \lim\limits_{x \to x_0} (f(x)-f(x_0)) = 0
        \end{align}
    }     
    \item Concept of `$f(x)\ dx$' when $x =x_0$ denotes {
        \begin{align}
            \lim\limits_{\Delta x \to 0} f(x_0)\ \Delta x
        \end{align}
    }  
    It means the idea is about limits and no one of $\Delta x$ and $dx$ do not
    depend to the $x_0$  and nor on the $f(x)$. So we do not write
    $\cancel{d(x)}$. 
\end{itemize}

\paragraph*{Note} For a continuous random variable, the CDF also is continuous.



\pagebreak

\section{Limit of continuous random variable}

For an arbitrary random variable $S$ we can say:

\begin{align}
    Pr(s_0 < S < s_1) &= CDF_s(s_1) - CDF_s(s_0) \nonumber \\ 
                      &= \int_{s_0}^{s_1} PDF_s(s) \ ds&&\\
               (1),(4)& \Rightarrow \exists s_2 \in [s_0,s_1] : Pr(s_0 < S < s_1) = PDF_s (s_2) \ (s_1 - s_0)\\
               (2),(5)& \Rightarrow \lim\limits_{s_1\to s_0} \Bigl(PDF_s(s_3)  \  (s_1 - s_0)\Bigr) = PDF_s(s_0) \  (s_1 - s_0) \\
               (3),(6)&  \Rightarrow \lim\limits_{s_1\to s_0} \Bigl(PDF_s(s_3)  \  (s_1 - s_0)\Bigr) = PDF_s(s_0) \  ds \nonumber
\end{align}
So we can say:
\begin{align}
    \lim\limits_{s_1\to s_0} \Bigl(CDF_s(s_1) - CDF_s(s_0)\Bigr) = PDF_s(s_0) \  ds
\end{align}

\pagebreak

\section{Monotonically increasing transformation}

\paragraph*{}{
    
    Suppose
    \begin{itemize}
        \item $s = T(r)$ is a monotonically increasing transformation from a random variable $r$ to another one $s$, so $T(r)$ is $1-1$.
        \item $s_0 = T(r_0), s_1 = T(r_1)$.
    \end{itemize}

    We know:
    \begin{align}
        CDF_s(s_1) - CDF_s(s_0) &= Pr(s_0 < S < s_1) \nonumber \\
        &= Pr\Bigl(T^{-1}(s_0) < T^{-1}(S) < T^{-1}(s_1)\Bigr) \\
        &=Pr(r_0 < R < r_1) \nonumber \\
        =CDF_r(r_1) - CDF_r(r_0)
    \end{align}
}
\paragraph*{Note} The eq.(8) is proved in \emph{measure theory}.

\paragraph*{}{
    \begin{align}
        (7),(9) \Rightarrow PDF_s(s_0) \ ds = PDF_r(r_0) \ dr
    \end{align}
}

\paragraph*{}{
    
Also you can see:
\begin{itemize}
    \item https://dsp.stackexchange.com/a/61112/11856
    \item https://dsp.stackexchange.com/q/30502/11856
\end{itemize}
}

\pagebreak

\section*{}

\paragraph*{}
Let 
\begin{itemize}
    \item $r$ is a \emph{continuous} random variable
    \item $p_r$ is $PDF_r$
    \item $s$ is another random variable
        \begin{align}
            s =T(r) = (L-1) \int_{0}^{r} p_r(w) \ dw
        \end{align}
    \item $p_s$ is $PDF_s$
\end{itemize}

Now we want to show that 
\begin{align*}
    p_s = \frac{1}{L-1}
\end{align*}

Really $T(r)$ is $CDF_r(r) - CDF_r(0)$
\footnote{This is true for continuous random variables. For a discrete random variable, $T(r)$ will be $CDF_r(r)$.}
, so it is a continuous and monotonically increasing function. So we can use
eq.(10): 
\[ p_s = p_r \frac{dr}{ds} \]

\paragraph*{}


\begin{proof}
    To use eq.(10), we only need to get the $ds$:
    \begin{align*}
        (11) \Rightarrow\\ 
        ds &= d(T(r)) \ dr = (L-1) p_r \  dr\\
        p_s &= p_r \frac{dr}{ds} && \text{replacing $ds$ in eq.(10)}\\
        &=p_r \frac{dr}{(L-1) p_r \  dr}\\
        &=\frac{1}{L-1}
    \end{align*}
\end{proof}

\pagebreak

\section{Summary}

\begin{itemize}
    \item preface
    \begin{itemize}
        \item relation between CDF and PDF
        \item CDF and PDF $d$
        \item measure theory and monotonically transformation $\Rightarrow$
        equal CDF s
        \item $p_s = \frac{dr}{ds}p_r$
    \end{itemize}
    \item $$T(r) = (L-1)\int_{0}^{r}p_r(w)\ dw$$
\end{itemize}
For a continuous random variable, $r$, above $T(r)$ will get a uniform random
variable $s$ which $p_s$ is $\frac{1}{L-1}$.
\paragraph*{Note} for histogram equalization, $L-1$ is max possible intensity
level of the range of image data type, or equivalently you can think $L$ is the
number of possible intensity levels in the image:
\begin{itemize}
    \item double $\rightarrow$ 1
    \item uint8 $\rightarrow$ 255
    \item uint16 $\rightarrow$ $2^{16} - 1$
\end{itemize}



\end{document}
