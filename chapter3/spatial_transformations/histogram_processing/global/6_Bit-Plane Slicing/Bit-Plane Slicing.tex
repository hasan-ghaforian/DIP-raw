\subsection{Bit-Plane Slicing}

The operation is:
\begin{enumerate}
    \item converting the pixel values in the binary form (\emph{sequence} of 
        bits)
    \item dividing bits into bit planes
\end{enumerate}

Notes:
\begin{itemize}
    \item Number of \emph{bit-planes} is the number of bits in each pixels.
    \item Size of \emph{bit-planes} must be same as size of image.
\end{itemize}

So for an 8bit image, you have to consider 8 planes and for 3bit image, 3 planes.

To get most significant bit (\emph{MSB}) and Least significant bit (\emph{LSB})
, convert data to binary. Left bit is \emph{MSB} and right bit is \emph{LSB}.
Left bit is called \emph{MSB} because if left bit of $11111111$ is changed to 
$0$ (i.e. $01111111$), then the value changes from $255$ to $127$. But if you 
change the right bit, it will change from $255$ to $254$.

In bit-plane slicing of an 8bit (\emph{uint8}) image, 8th plane is plane of 
\emph{MSB}s.

\subsubsection{Transformations}

Transformation for getting nth plane from decimal number
    integrally divide number to $2^{n-1}$ get the mode of result to the $2$

Transformation for getting $n$th plane from binary number
    bit shifting 

compositing of bit-planes
    for ith plane mutiply its element by $2^{i-1}$
    pluse all multiplied planes

number of colors in each bit plane: $2$
number of colors by compositing m bit-plane: $2^m$

\subsubsection{Octave lab}
