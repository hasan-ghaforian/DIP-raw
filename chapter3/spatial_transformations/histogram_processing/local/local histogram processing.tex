
Result for each pixel in these ways does not depend on doing globally or
locally: 
\begin{itemize}
    \item negative
    \item gamma
    \item logarithmic
    \item Intensity Level Slicing
    \item Bit-Plane Slicing
\end{itemize}


Above is true while you do not compute their arguments based on set of pixels.
For example if you select $\sigma$ of pixels as argument for \emph{gamma} 
correction, then its value may be different while you calculate it for all 
pixels of image or an area of it.

For \emph{contrast stretching} when we use percentage thresholding, its local 
and global may differ.

Approaches for local histogram processing (Gonzalez pages 149,150):
\begin{enumerate}
    \item The procedure is to define a neighborhood and move its center from pixel to
    pixel in a horizontal or vertical direction. At each location, the histogram
    of the points in the neighborhood is computed, and either a histogram
    equalization or histogram specification transformation function is obtained.
    This function is used to map the intensity of the pixel centered in the
    neighborhood. The center of the neighborhood is then moved to an adjacent
    pixel location and the procedure is repeated.

    \item Another approach used sometimes to reduce computation is to utilize 
    nonoverlapping regions, but this method usually produces an undesirable
    “blocky” effect.
\end{enumerate}

For the first approach:
    \begin{quotation}
        Because only one row or column of the neighborhood changes in a one-pixel
        translation of the neighborhood, updating the histogram obtained in the
        previous location with the new data introduced at each motion step is
        possible (see Problem 3.14). This approach has obvious advantages over
        repeatedly computing the histogram of all pixels in the neighborhood region
        each time the region is moved one pixel location. 
        For this technique, see https://dsp.stackexchange.com/a/67274/11856.
    \end{quotation}
